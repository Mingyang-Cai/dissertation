\chapter*{Samenvatting}
\addcontentsline{toc}{chapter}{Summary in Dutch / Samenvatting}
\markboth{Samenvatting}{Samenvatting}


Joint modeling (JM) en fully conditional specification (FCS) zijn twee veelgebruikte strategieën om multivariate incomplete data te imputeren. JM bestaat uit het specificeren van een multivariate verdeling voor de ontbrekende data om vervolgens imputaties uit de betreffende conditionele verdelingen te trekken. Bij FCS specificeert men de verdeling van iedere gedeeltelijk geobserveerde variabele conditioneel op alle andere variabelen. FCS heeft als voordeel boven JM dat het modeleren van multivariate data zeer flexibel kan worden vormgegeven. Het komt echter in de praktijk met regelmaat voor dat bepaalde structuren in de ontbrekende gegevens niet adequaat gemodelleerd kunnen worden. Daarnaast is het vaak lastig om de relaties tussen meerdere variabelen te bewaren wanneer een stapsgewijze imputatietechniek als FCS wordt gebruikt. Deze dissertatie heeft als doel om hybride imputatiestrategieën te ontwikkelen waarin aantrekkelijke eigenschappen van JM en FCS worden samengenomen. In dit proefschrift draag ik oplossingen aan voor ontbrekende dataproblemen waarin het toepassen van enkel JM of FCS tot suboptimale oplossingen leidt. 

In Hoofdstuk 2 beschouw ik algemene methoden om gekwadarateerde relaties te imputeren. Ik verbeter de polynomial combination (PC) methode en evalueer deze verbetering tegen oplossingen verkregen middels het Substantive Model Compatible Fully Conditional Specification (SMCFCS) framework. De uitkomst van deze evaluatie is als volgt: Als de ware verdeling en model bekend zijn, dan levert SMCFCS de scherpste inferentie. Dit gaat wel ten koste van de flexibiliteit in modelleren. Met de aangepaste PC methode blijft deze flexibiliteit wél behouden, maar kan de inferentie wat minder scherp zijn in situaties waar de verdeling van de geobserveerde data soms spaars is.

In Hoofdstuk 3 ontwikkel ik Multivariate Predictive Mean Matching (MPMM). Met MPMM kan men meer dan één incomplete variabele tegelijk imputeren. Ik combineer de methodologie van univariate PMM (REF LITTLE REF RUBIN) en canonical regression analysis (REF) om de ruimtes van de voorspellers en uitkomsten samen te vatten. Het voordeel van deze imputatiemethode is dat de relaties tussen incomplete variabelen behouden kunnen worden. Ik behandel een aantal scenario’s waarin MPMM gebruikt kan worden en beschouw de beperkingen van deze nieuwe methode. 

In Hoofdstuk 4 ontwikkel ik een hybride imputatietechniek om individuele behandeleffecten te kunnen schatten. Het imputeren van niet-geobserveerde uitkomsten maakt het mogelijk om verschillen tussen potentiele uitkomsten te berekenen voor verschillende behandelcondities. Het fundamentele probleem hierbij is dat beide uitkomsten nooit tegelijk geobserveerd kunnen worden. Men moet dus aannames doen over de correlatie tussen de potentiele uitkomsten. De voorgestelde hybride imputatiemethode stelt ons in staat om de partiële correlatie te specificeren en een sensitiviteitsanalyse uit te voeren. Ik demonstreer de validiteit van de voorgestelde hybride imputatiemethode en laat zien hoe men deze techniek in de praktijk kan toepassen. 
 
In Hoofdstuk 5 onderzoek ik de compatibiliteit van FCS onder informatieve prior verdelingen. Dergelijke vergelijkingen zijn al door meerdere auteurs beschreven voor scenarios waarin de prior voor de conditionele modellen niet informatief is. Compatibiliteit eigenschappen voor informatieve priors zijn echter onderbelicht gebleven. Ik toon aan dat FCS onder het normale lineaire model met een informatieve inverse-gamma prior compatibel is met een gezamenlijke verdeling en lever de bijbehorende normale inverse-Wishart prior verdeling voor de gezamenlijke verdeling.  

In Hoofdstuk 6 ontwikkel ik een nieuwe strategie voor het diagnosticeren en evalueren van multiple imputation modellen door middel van posterior predictive checking. Het idee hierbij is dat, wanneer het imputatiemodel congenial (REF MENG) is met betrekking tot het inhoudelijke analysemodel, de verwachte waarde van de geobserveerde data zich in het midden van de bijbehorende posterior predictive verdeling bevindt. Door de voorgestelde diagnostische methode te gebruiken, kunnen onderzoekers de geobserveerde data vergelijken met de ‘overgeïmputeerde’ data om de fit van het imputatiemodel te evalueren. 
 
